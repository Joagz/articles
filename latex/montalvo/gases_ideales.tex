\documentclass{article}
\RequirePackage{amsfonts}
\usepackage[spanish]{babel}
\usepackage[inkscapeformat=png]{svg}

\title{Calculo de presión de un gas ideal}
\author{Joaquín Gómez}
\begin{document}
\maketitle

\section*{Derivación de una fórmula para la presión de un gas ideal}

Antes de empezar, derivamos una ecuación a partir de un experimento: suponemos
tener un recipiente, que cuenta con un pistón en uno de sus lados.

\begin{figure}[h]
  \centering
  \caption{Recipiente de volumen $V$ variable}
  \includesvg{piston}
\end{figure}

También, decimos que el pistón no deja escapar ninguna molécula del gas, y que
no tiene rozamiento con las paredes del contenedor.

La presión del gas en el recipiente se puede modelar con una ecuación

\[
  P = \frac F A
\]

Esto quiere decir que a mayor sea el área del pistón, la presión será menor
(esto es equivalente a decir que, si disminuimos el área del pistón,
disminuimos el volumen del contenedor. Por lo tanto la presión aumenta, ya que
tenemos la misma cantidad de gas y la misma temperatura, en un volumen
reducido).

Al comprimir el gas estamos realizando un trabajo sobre él. Al hacer mover las
partículas en un instante infinitesimal de tiempo, el trabajo instantáneo $dW$
sobre el gas está dado por $dW=Fdx$.

Ahora, este trabajo que realizamos puede reescribirse como $dW=ma\cdot dx$, a
su vez, si reescribimos la aceleración como $a = \frac {dv}{dt}$, observamos
que

\[
  dW = m \frac{dv}{dt} dx = m \frac{dx}{dt}dv = mv \cdot dv
\]

Tomando la integral con respecto de la velocidad, obtenemos la fórmula de
Energía Cinética

\[
  W = \int {mv\,dv}= \frac 1 2 m v^2
\]

De forma alternativa, podemos ver que nuestra fórmula para la presión nos da
$F=PA$, de modo que el trabajo en un instante infinitesimal de tiempo es

\[
  dW=Fdx=PAdx
\]

pero $Adx$ es el volumen de una ``lámina'' del área del pistón (no es preciso
llamarlo así, también podemos pensarlo como el volumen de una porción
infinitesimal del recipiente, si lo cortaramos de forma transversal). De esta
manera podemos decir que el cambio en el volumen es $dV=Adx$. A medida que
efectuamos presión con el pistón, el volumen disminuye, por lo tanto la tasa de
cambio del volumen debe ser contraria a la del trabajo. Como resultado
obtenemos

\[
  dW = -PdV
\]

Pensemos ahora en la presión que ejercen las moléculas en el pistón. Cada vez
que una partícula choca contra el pistón, ésta genera un momento que desplaza
el pistón hacia afuera. Debemos encontrar una fuerza $F$ que iguale la fuerza
total de las partículas chocando contra el pistón.

En un instante de tiempo $dt$, la partícula del gas chocará contra la pared del
pistón, la distancia recorrida por la partícula está dada por $v\, dt$. Las
partículas que choquen contra el pistón estarán contenidas en un volumen dado
por $vA\,dt$, definimos además que $n$ es el número de átomos por unidad de
volumen, de forma que obtenemos $nvA\,dt$.

Como $t$ está dado en segundos, y nosotros queremos la cantidad de partículas
por segundo que impactan en el pistón, ponemos $t=1$ y obtenemos $nvA$.

Ahora, el momento que ejerce la partícula está dado por $mv$, donde $m$ es la
masa de la partícula. Si se tiene en cuenta que la molecula tiene dos momentos,
uno de entrada y otro de salida (antes del impacto y luego de éste), entonces
el momento total está dado por $2mv$. (Aclaración, $v$ es la velocidad en
dirección al pistón).

La fuerza entonces está dada por
\[
  F = nvA\cdot2mv = 2nmv^2 A
\]

Teniendo en cuenta que $P=F/A$ podemos escribir

\[
  P = 2nmv^2
\]

Sin embargo, hay un problema con esta fórmula. La velocidad de la molécula es
un vector $\vec{v}\in\mathbb R^3$. Esto quiere decir que la velocidad está dada
en tres ejes. Además, las velocidades son aleatorias. Para tener en cuenta
esto, lo que haremos será calcular la media cuadrática de la velocidad. Esta
medición (a diferencia de la media aritmética) nos permite tener en cuenta la
velocidad en direcciones opuestas (ya que los átomos se desplazan en todas las
direcciones), es decir, si el átomo se desplaza en el eje $x$ en dirección
negativa o positiva, lo que sucedería con la media aritmética es que los
términos se restarían, y habría un error en la medición.

La media cuadrática está dada por $RMS = \sqrt{\frac 1 n \sum_{k=1}^{n}x_k^2}$.
Aplicando esto a nuestro vector vemos que

\[
  v_{rms}^2 = \langle {v}^2 \rangle =  \frac{1}{3}\left(v_x^2+v_y^2+v_z^2\right) = \frac{1}{3} |\vec{v}|^2,
\]

y suponemos que el promedio de velocidades es igual en todas las direcciones

\[
  \langle v_x^2 \rangle = \langle v_y^2 \rangle = \langle v_z^2 \rangle
\]

Si suponemos que la mitad de las partículas se desplazan en una dirección, y al
mitad en la otra dirección, podemos escribir

De modo que reescribimos la fórmula de la presión utilizando la velocidad
promedio

\[
  P = \frac{1}{3}nmv_{rms}^2
\]

Teniendo en cuenta que la energía cinética está dada por $E_c = \frac 1 2 m
  v^2$ podemos reescribir $P$ como

\[
  P = \frac{2}{3}n\frac{1}{2}mv_{rms}^2=\frac{2}{3}nE_c
\]

Dado que $n$ es la cantidad de moléculas por unidad de volumen, es decir,
$n=N/V$, se tiene

\[
  P = \frac{2}{3}\frac{N}{V}E_c
\]

O, de forma alternativa: $PV = \frac{2}{3}NE_c=\frac{1}{3}Nmv_{rms}^2$.

\section*{Ejemplo: calcular la presión de un gas ideal}

Supongamos que tenemos un gas ideal con un volumen de 1 m³. Sabemos que la
velocidad cuadrática media de las moléculas de gas es
$v_{rms}=500\left[\frac{m}{s}\right]$. Queremos calcular la presión del gas a
una temperatura de $T=300 [K]$ y dado que la masa de cada molécula de gas es
$m=4.65\times10^{-26} [kg]$.

\subsection*{Solución}

Con la fórmula de la presión

\[
  P = \frac{2}{3}\frac{N}{V}E_c
\]

Podemos ver que tenemos el volumen $V$ del gas, la velocidad media $v_{rms}^2$,
la temperatura $T$ y la masa $m$ de cada partícula. De forma que obtenemos

\[
P=\frac{\frac{2}{3}Nmv_{rms}^2}{V}=\frac{\frac{2}{3}N}{1}
= \frac{1000}{3} \cdot(4.65\times10^{-26})\cdot N
\]

Lo que faltaría sería el número de partículas en el recipiente. (?)

\end{document}
