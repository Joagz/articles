\documentclass[./my-proofs.tex]{subfiles}

\begin{document}
\section{Real Analysis}

\subsection{De Morgan's Laws: Finite and Infinite Cases}
\begin{enumerate}
    \item Show how induction can be used to conclude that
          \begin{equation}
              {\left(A_1\cup A_2 \cup \cdots \cup A_n\right)}^{c} =A_1^{c} \cap A_2^{c} \cap \cdots \cap A_n^{c}
          \end{equation}
          for any finite $n\in\mathbb{N}$.

    \item Conclude that
          \begin{equation}
              {\left(\bigcup^{\infty}_{i=1}{A_i}\right)}^{c} = \bigcap^{\infty}_{i=1}{A_i^{c}}
          \end{equation}
\end{enumerate}

\begin{enumerate}
    \item Morgan's Laws state that $(A\cap B)^{c} = A^{c} \cup B^{c}$ and ${(A \cup B)}^{c}
              = A^{c} \cap B^{c}$, if we suppose that the initial case is true, suppose that
          it is true for $n$ and for $n+1$ we have
          \begin{equation}
              \begin{split}
                  {\left(\left(A_1\cup \cdots \cup A_n\right)\cup A_{n+1}\right)}^{c} = {\left(A_1\cup \cdots \cup A_n\right)}^{c}\cap A_{n+1}^{c} \\
                  =  \left(A_1^{c} \cap \cdots \cap A_n^{c}\right)\cap A_{n+1}^{c}\\
                  =  A_1^{c} \cap \cdots \cap A_n^{c}\cap A_{n+1}^{c}
              \end{split}
          \end{equation}
          Which proves the statement.

    \item We will prove it for both sides, first we can look at the following logic and
          apply it to the proof: \textit{If $x\in A$ implies that $x\in B$, then
              $B\subseteq A$. On the other hand, if $x\in B$ implies that $x\in A$, then
              $B\supseteq A$}. To put it in words, if $x$ is part of $A$ means that it is
          \textit{always true} that $x\in B$, then $B$ is contained in $A$, but in the
          other way, $x$ being in $B$ does not imply that $x$ is in $A$. To prove
          equivalence, both statements should be true, this is, $B\subseteq A$ and
          $B\supseteq A$.

          \begin{itemize}
              \item[($\subseteq$)] If $x\in\bigcap^{\infty}_{i=1}{A_i^{c}}$, then $x\in A_i^{c}$ for all $i\in\mathbb{N}$.
                  This implies that $x\notin A_i$ for all $i\in\mathbb{N}$, thus $x\notin{\bigcup^{\infty}_{i=1}{A_i}}$, which is the
                  same as $x\in{\left(\bigcup^{\infty}_{i=1}{A_i}\right)}^{c}$. So we have
                  \begin{equation}
                      {\left(\bigcup^{\infty}_{i=1}{A_i}\right)}^{c} \subseteq \bigcap^{\infty}_{i=1}{A_i^{c}}
                  \end{equation}

              \item[($\supseteq$)] If $x\in{\left(\bigcup^{\infty}_{i=1}{A_i}\right)}^{c}$, then $x\notin A_i$ for all $i\in\mathbb{N}$.
                  This is the same as $x\in A_i^{c}$ for all $i\in\mathbb{N}$. Thus $x\in\bigcap^{\infty}_{i=1}{A_i^{c}}$. We conclude that
                  \begin{equation}
                      {\left(\bigcup^{\infty}_{i=1}{A_i}\right)}^{c} \supseteq \bigcap^{\infty}_{i=1}{A_i^{c}}
                  \end{equation}

          \end{itemize}
\end{enumerate}

\end{document}