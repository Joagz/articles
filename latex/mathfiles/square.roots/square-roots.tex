\documentclass{article}

\title{Proof of existence of square roots}
\author{Joaquín Gómez}
\usepackage{amsfonts}
\usepackage{amsmath}

\begin{document}

\maketitle

Let $T$ be a subset of the real numbers given by

\[
    T=\left\{
    t\in \mathbb R : t^2 < 2
    \right\}
\]

We want to prove that the supremum of this set, which should be irrational,
exists. We define $\alpha=\sup T$. We want to get that $\alpha^2 = 2$ by ruling
out the posibilities of $\alpha^{2} < 2$ and $\alpha^{2} > 2$

First, imagine a line of real numbers, we can try to push the supremum by a
factor of $1/n$

\begin{align*}
    \left(\alpha + \frac{1}{n}\right)^2 = \alpha^{2} + \frac{2\alpha}{n} + \frac{1}{n^{2}}
    \\< \alpha^{2} + \frac{2\alpha}{n} + \frac{1}{n}
    \\=\alpha^{2} + \frac{2\alpha + 1}{n}
\end{align*}

We now assume that there's some value of $n$ such that

\begin{align*}
    \alpha^{2} + \frac{2\alpha + 1}{n} < 2
\end{align*}

After a rearangement we get

\begin{align*}
    \frac{2\alpha + 1}{n} > 2 - \alpha^{2}
\end{align*}

Now, we can get back to the first inequality. Notice that

\begin{align*}
    \left(\alpha + \frac{1}{n}\right)^2 < \alpha^{2} + (2-\alpha^{2}) = 2
\end{align*}

This contradicts the fact that $\alpha = \sup T$. Thus, it means that we still
can ``move towards the right'' after $\alpha$ and get a number in $T$.

Now, we look to a contradiction for $\alpha^{2} > 2$. Assuming this is true, we
could subtract some amount to $\alpha$ and get a number in $T$.

\begin{align*}
    \left(\alpha - \frac{1}{n}\right)^{2} = \alpha^{2} - \frac{2\alpha}{n} + \frac{1}{n^{2}}
    \\>\alpha^{2} - \frac{2\alpha}{n}
\end{align*}

Now, analogous to the first part of the Proof

\begin{align*}
    \alpha^{2} - \frac{2\alpha}{n} > 2
\end{align*}

Rearranging the inequality we get $2-\alpha^{2} < \frac{-2\alpha}{n}$, this
means that

\begin{align*}
    \left(\alpha - \frac{1}{n}\right)^{2} > \alpha^{2} + (2-\alpha^{2}) = 2
\end{align*}

This means that $\alpha^{2} > 2$ is false.

And because both $\alpha^{2} < 2$ and $\alpha^{2} > 2$ are false, this means
that $\alpha^{2} = 2$, which proves the existence of square roots in $\mathbb{R}$.

\end{document}
