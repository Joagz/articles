\documentclass[../linear-spaces.tex]{subfiles}

\begin{document}

\chapter{Differentiation}

This chapter will go through a different topic, in this case we will explore differentiation.
However, we will assume the reader has knowledge on one-dimensional differentiation, and some
theory, to explore multi-dimensional derivatives, in vector fields. We will ignore scalar fields 
because they are, in some way, a subset of vector fields.

First, let's introduce the concept of ``derivative'' for a scalar field. Let $f: \mathbb{R}^{n} \to \mathbb{R}^{m}$,
we would need a set of functions $f_1,f_2,\dots,f_m: \mathbb{R}^{n} \to \mathbb{R}$ to describe the 
derivative in terms of scalar fields. This leads to the conclussion that, given that we have a lot of variables, we 
can certainly form a vector for each $f_i$ so that 
\begin{equation*}
    \nabla f_i = 
    \begin{bmatrix} 
        \dfrac{\partial f_i}{\partial x_1} & \cdots & \dfrac{\partial f_i}{\partial x_n} 
    \end{bmatrix}
\end{equation*}

This vector is called the gradient of $f$. The gradient has various properties that we will not describe or talk about right now. 
However, it is important to note that this is a vector that \textbf{points to the steepest direction} on the scalar field. Imagine 
a cannon or vulcano, the gradient in this figure will always point upwards.

In other words, the value of this vector at a certain point $p$ gives the direction and the rate of fastest increase.

Also, we can demonstrate (we will not for now, because we need the chain rule to do so) that the gradient is perpendicular to the 
level curves of the scalar field. Let $f: \mathbb{R}^{2} \to \mathbb{R}$, we can only provide this example in three dimensions, 
in other way it would be impossible to imagine. A level curve for $f$ will be, for example 
\begin{equation}
    f(x,y)=k\quad \text{for some } k \in \mathbb{R}
\end{equation}

Then, the gradient $\nabla f = \begin{bmatrix}  \dfrac{\partial f}{\partial x} &  \dfrac{\partial f}{\partial y} \end{bmatrix}$ will 
satisfy $\nabla f \cdot T = 0$, where $T$ is the tangent vector to the level curve. This vector is obtained by parametrisation on 
$f$. We can let $r(t)$ defined in $[a,b]$, and let $f(x,y) = f\left[X(t), Y(t)\right] = f\left[r(t)\right] = g(t)$, by the chain rule 
we have 
\begin{equation}
    g'(t) = f'\left[r(t)\right]\cdot r'(t)
\end{equation}
and we know that $r'(t)$ is tangent to the path of $r(t)$. Given that the path is a level curve of $f$, we would have $g(t) = k$, because 
$f(x,y) = k$ for a level curve, we would get 
 \begin{equation}
    f'\left[r(t)\right]\cdot r'(t) = 0
\end{equation}
which implies that $f'\left[r(t)\right] \perp r'(t)$. Now, to prove that 
\begin{equation}
    f'\left[r(t)\right] = \nabla f[r(t)]
\end{equation}
we would need to write about Taylor's approximation theorem. For the moment, we will keep that the equality as true. So we now get 
\begin{equation}
    \nabla f \left[r(t)\right] \cdot r'(t) = 0
\end{equation} 



\end{document}