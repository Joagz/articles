\documentclass[../linear-spaces.tex]{subfiles}

\begin{document}

\chapter{Differentiation}

In this chapter we will explore differentiation. However, I will assume the
reader has knowledge
of one-dimensional differentiation. We'll explore multi-dimensional derivatives
in scalar and vector fields.

\begin{definition} (Scalar field)
    A scalar field is a mapping $f:\mathbb{R}^{n} \to \mathbb{R}$ for $n\in
        \mathbb{N}$.
\end{definition}

\begin{definition} (Vector field)
    A vector field is a mapping $f:\mathbb{R}^{n} \to \mathbb{R}^{m}$ for
    $n,m\in \mathbb{N}$.
\end{definition}

Let's first introduce the concept of ``derivative'' for a vector field. Let $f:
    \mathbb{R}^{n} \to \mathbb{R}^{m}$,
we would need a set of functions $f_1,f_2,\dots,f_m: \mathbb{R}^{n} \to
    \mathbb{R}$ to describe the
derivative in terms of scalar fields. For each $f_i$, we can form a vector of
partial derivatives
\begin{equation*}
    \nabla f_i =
    \begin{bmatrix}
        \dfrac{\partial f_i}{\partial x_1} & \cdots & \dfrac{\partial
            f_i}{\partial x_n}
    \end{bmatrix}
\end{equation*}

This vector is called the gradient of $f$. The gradient has various properties
that we will not describe or talk about right now.
However, it is important to note that it is a vector that \textbf{points to the
    steepest direction} on the scalar field, and
always points upwards\footnote{This has to do with the definition of the
    derivative, visualize in 3-dimensions
    a surface and put a tangent line on it. What can you say about the slope of
    the tangent? We define it is always pointing upwards,
    so the tangent line has positive slope.}.

In other words, the value of this vector at a certain point $p$ gives the
direction and the rate of fastest increase.

Also, we can demonstrate (we will not for now, because we need the chain rule
to do so) that the gradient is perpendicular to the
level curves of the scalar field. Let $f: \mathbb{R}^{2} \to \mathbb{R}$, we
can only provide this example in three dimensions,
in other way it would be impossible to imagine. A level curve for $f$ will be,
for example
\begin{equation}
    f(x,y)=k\quad \text{for some } k \in \mathbb{R}
\end{equation}
Then, the gradient $\nabla f = \begin{bmatrix}	\dfrac{\partial f}{\partial x}
         & \dfrac{\partial f}{\partial y}\end{bmatrix}$ will
satisfy $\nabla f \cdot T = 0$, where $T$ is the tangent vector to the level
curve. This vector is obtained by parametrisation on
$f$. We can let $r(t)$ defined in $[a,b]$, and let $f(x,y) = f\left[X(t),
        Y(t)\right] = f\left[r(t)\right] = g(t)$, by the chain rule
we have
\begin{equation}
    g'(t) = f'\left[r(t)\right]\cdot r'(t)
\end{equation}
and we know that $r'(t)$ is tangent to the path of $r(t)$. Given that the path
is a level curve of $f$, we would have $g(t) = k$, because
$f(x,y) = k$, we get
\begin{equation}
    f'\left[r(t)\right]\cdot r'(t) = 0
\end{equation}
which implies that $f'\left[r(t)\right] \perp r'(t)$. Now, we have to prove
that
\begin{equation}
    f'\left[r(t)\right] = \nabla f[r(t)]
\end{equation}

\section{Directional derivative}

The derivative in a scalar field depends from the direction that we choose. We
define
the \textit{directional derivative} of $f(x)$ with respect to the vector $a$ as
\begin{equation}
    f'(x; a) = \lim_{h\to 0}\dfrac{f(x + ha) - f(x)}{h}
\end{equation}
such that, for sufficiently small $h$ we have
\begin{equation}
    f(x) \approx f(x+ha) - h f(x; a)
\end{equation}

The directional derivatives in the direction of the basis vectors is a set
$S = \left\{f(x; e_1), f(x; e_2), \dots,f(x; e_n)\right\}$. These derivatives
are
the partial derivatives of $f$, so
\begin{equation}
    f'(x; e_k) = \dfrac{\partial f}{\partial x_k}
\end{equation}
We can express $f(x; a)$ using linearity
\begin{equation}
    f'(x; a) = f'\left(x; \sum_{i=1}^{n}{a_i e_i} \right)
    = \sum_{i=1}^{n}a_i f'\left(x; e_i\right) =
    \sum_{i=1}^{n}a_i \dfrac{\partial f}{\partial x_k}
\end{equation}
this is equal to the dot product
\begin{equation}
    f'(x; a) = \nabla f(x) \cdot a
\end{equation}
We can now rewrite (5.7) as
\begin{equation}
    f(x) \approx f(x+ha) - h\nabla f(x)\cdot a
\end{equation}
we can also set $a$ to be a very small vector, call it $\delta x$, which leads
to
\begin{equation}
    f(x) \approx f(x+\delta x) - \nabla f(x)\cdot \delta x
\end{equation}
This is called a \textit{first order Taylor's approximation}.
\begin{equation}
    f(x) \approx f(x+\delta x) - \sum_{k=1}^{n} \dfrac{\partial f}{\partial x_k}\cdot \delta x
\end{equation}

\section{Differentials in vector fields}
We have already defined a vector field, let $f:\mathbb{R}^{n} \to \mathbb{R}^{m}$.
We can represent this transformation as a system. Also, let $x=(x_1,\dots,x_n)$, and
\begin{equation}
    f({x})=\begin{cases}
        f_1(x_1,\dots,x_n) \\
        f_2(x_1,\dots,x_n) \\
        \vdots             \\
        f_m(x_1,\dots,x_n) \\
    \end{cases}
\end{equation}
so we can define $f({x}) = \left[f_1({x}), \dots,f_m({x})\right]$
In terms of scalar fields, we have
\begin{equation}
    \nabla f_i({x}) = \left[\dfrac{\partial f_i}{\partial x_1} \cdots \dfrac{\partial f_1}{\partial x_n}\right]
\end{equation}
We can use a first order Taylor's approximation to approximate $f(x)$, using (5.11) we have
\begin{equation}
    f(x) \approx f(x+\delta x) - \nabla f(x)\cdot \delta x
\end{equation}
We can write the full approximation as a sum of two matrices
\begin{equation}
    \begin{split}
        f(x) &\approx \begin{bmatrix}
            f_1(x+\delta x) \\
            f_2(x+\delta x) \\
            \vdots          \\
            f_m(x+\delta x) \\
        \end{bmatrix} + \begin{bmatrix}
            \nabla f_1(x)\cdot\delta x \\
            \nabla f_2(x)\cdot\delta x \\
            \vdots                     \\
            \nabla f_m(x)\cdot\delta x \\
        \end{bmatrix} \\[1em]
        &= f(x+\delta x) + D[f(x)]
    \end{split}
\end{equation}
where $D[f(x)]$ is called the \textbf{Jacobian matrix} of $f(x)$
\begin{equation}
    D[f(x)] = \begin{bmatrix}
        \dfrac{\partial f_1}{\partial x_1} & \dfrac{\partial f_1}{\partial x_2} & \cdots & \dfrac{\partial f_1}{\partial x_n} \\[1em]
        \dfrac{\partial f_2}{\partial x_1} & \dfrac{\partial f_2}{\partial x_2} & \cdots & \dfrac{\partial f_2}{\partial x_n} \\[1em]
        \vdots                             & \vdots                             & \ddots & \vdots                             \\[1em]
        \dfrac{\partial f_m}{\partial x_1} & \dfrac{\partial f_m}{\partial x_2} & \cdots & \dfrac{\partial f_m}{\partial x_n} \\[1em]
    \end{bmatrix}
\end{equation}
\end{document}