\documentclass[a4paper,12pt]{article}
\usepackage{amssymb}
\usepackage{amsthm}

\title{Consequences of the Axiom of Completeness}
\author{Joaquín Gómez}

% \theoremstyle{definition}
\newtheorem{theorem}{Theorem}

\begin{document}
\maketitle

\section{Introduction}
The Axiom of Completeness is an assumption made about real numbers, we know
that the numbering system is composed of natural, integers and rational
numbers. We define the last set to contain both naturals and integers and write
$\mathbb{N} \subseteq \mathbb{Z} \subseteq \mathbb{Q}$. However, a Pythagorean
cult member called Hippasus of Metapontum discovered an important fact about
numbers.

\section{$\sqrt{2}$ is irrational, a hole in $\mathbb{Q}$}

Our initial assumption is that $\sqrt 2$ can be written in the form $\frac p q$
where $p,q\in \mathbb{Z}$ and both have no common factor
\[
  {\left(\frac{p}{q}\right)}^2 = 2
\]
This implies
\[
  p^2 = 2q^2
\]
From this, we can see that $p^2$ is even, this means that $p$ is even too (the
square of an odd number is odd). We can write $p$ as
\[
  p=2r, \qquad r\in\mathbb{Z}
\]
Substitute this in the equation above and get
\[
  q^2=2r^2
\]
Thus, this implies that $q$ is also even, this is absurd because we have
initially assumed them to have no common factor. But as they have, we can
rewritte the fraction in lower terms, however, this process would be infinite.
Thus $\sqrt 2$ is not a rational number.

The existence of this proof demonstrates that there is a \textit{hole} in the
rational numbers. This set does not contain numbers such as $\sqrt 2, \sqrt 3,
  \pi$, etc. This problem lead mathematicians to build a set that includes
irrational numbers. This set is known as the real numbers.

\section{Proof: $\sqrt 3$ is irrational}

To prove this, we will first prove the following statement

\begin{theorem}
  If the square of a number is divisible by a prime, then the number itself is divisible by that prime
\end{theorem}

\begin{proof}
  Let $n$ be any number. Then $n$ has at least a prime factor $p$, then

  \[
    n^2=pk\qquad {k\in\mathbb Z}
  \]

  for some $k$. Now, we know that we can decompose $n$ into prime factors, let
  $p_1, p_2, \dots, p_n$ be the prime factors of $n$, such that

  \[
    n=p_1^{e_1}p_2^{e_2}\cdots p_n^{e_n}
  \]

  Thus $n^2=p_1^{2e_1}p_2^{2e_2}\cdots p_n^{2e_n}$, from this, we see that it is
  divisible by $p$, since $p\in\left\{p_1,p_2,\dots,p_n\right\}$.
\end{proof}

Now, we have to prove that $\sqrt{3}$. Analogous to the proof for $\sqrt 2$, we
let $p,q \in \mathbb Z$ be two numbers without any common factor. We try to
prove the statement by contradiction, we set

\[
  {\left(\frac p q\right)}^2 = 3
\]
Then $p^2=3q^2$, but by the theorem 1, we have that if $p^2$ is divisible by 3,
then $p$ must. So $p=3r$, then

\[
  3r^2 = q^2
\]
This means that $q$ is also divisible by 3. Then $p$ and $q$ have a common
prime factor, this violates the initial constraint for $p$ and $q$ to not have
any common factor.

\end{document}
